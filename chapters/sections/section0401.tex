%!TEX ROOT=main.tex

\section{Overall description}\label{sec:overall-description}

Based on the research and analysis introduced in the previous chapters, following specification will describe an application that fits the criteria.
Emphasised words are entities that are present in the domain model and are going to be revealed in the following section.

The application manages two types of personal development activities: goals and habits.
\textit{Goal} is an activity with defined start and finish.
It consists of \textit{tasks}, which makes it possible to track the progress on a goal.
Goal is built from \textit{goal plan}, which allows separation of goal creation from fulfillment.
That opens the possibility to share instructions on how to achieve the goal.
When applicable, the goal plan might be created by a person experienced in the field related to the goal.
\textit{Habit} is a repetitive activity.
It does not require decomposition to smaller tasks.
Habit is fulfilled according to a schedule.
Described separation to habits and goals allows users to select better data management form for their use case, which would usually require two different applications.

As it was mentioned many times, the application will engage users through theirs close social circle.
In order to do so, there are goal and habit related \textit{activities}.
Activity contains text with relevant information on habit streak or goal's progress.
Visibility of activities between users will be driven by habit and goal settings, they might be private, shared between all friends or just with selected users.
In order to provide simple communication tool, it is possible to leave \textit{commentaries} under activities.

\textit{Daily progress} on goals and habits is gathered in order to provide user's with statistics.
It is also used for weekly and monthly statistics calculation.
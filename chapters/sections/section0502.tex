%!TEX ROOT=main.tex

\section{Web client}\label{sec:web-client}

The most useful resources for the web client development were documentations for React and Material-UI.\cite{react, material}

Web client serves as a representational layer over server endpoints.
React web applications are built based on a single-page application (SPA) model.
The main idea of SPA is to have one HTML page that will dynamically change its content using JavaScript.
Benefits of this model are faster transitions and more fluent feel overall.
Downside of SPA is more complicated search engine optimization.
Search engines web crawlers were historically designed to work with HTML served content, but with SPA content is served with JavaScript.

In React application there is one HTML file where the main "App" component is placed.
Using React Router different page components might be served.
Page components are built by smaller reusable components, e.g.\ list of habits for today might be reused for upcoming habits.
These reusable components might be also reused in mobile application prototyping.

The implemented client was ad-hoc tested during the development against local instance of the server.
Later it was tested by users.

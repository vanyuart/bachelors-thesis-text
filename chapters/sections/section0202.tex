%!TEX ROOT=main.tex

\section{Competitive attitude}\label{sec:competitive-attitude}

%https://www.frontiersin.org/articles/10.3389/fpsyg.2018.00779/full

%Regarding the personal-development competitive attitude (PDCA),
%the primary focus is on personal growth and on the enjoyment and mastery of the task in a competitive situation.
%The goal attainment and competition outcome (i.e., on winning) is important,
%but not at the expense of the derogation of other competitors (Ryckman et al., 1996).

Whereas the previous section created a basis for data management and data flow,
this section will introduce justification for competitiveness as an instrument for personal development.
Social engagement element of the application will be based on this justification.

For a long time, until the 1990s, in psychology competitiveness was considered as a unidimensional scale ~\cite{the-four-faces-of-competetition}.
Cooperation could be seen as the opposite of competitiveness.
A research article "The Four Faces of Competition" summarized the development and research of recent decades and introduced four-dimensional scale for competitiveness.
This scale heavily based on works of R.M.Ryckman ~\cite{ryckman-pdca}~\cite{ryckman-hca}~\cite{ryckman-caa}.
R.M.Ryckman studied competitiveness as attitude rather than personal quality.
Overall there were identified four following attitudes regarding competition:
hypercompetitive, anxiety-driven competition avoidant, self-developmental,  and lack of interest toward competition.

\textit{Hypercompetitive} attitude is represented by aggressive approach to competitive situations.
One with hypercompetitive attitude has a goal of winning any competition they participate.
Winning could be gained even by unfair means.
This dominant attitude is destructive for personal relationships and considered psychologically unhealthy ~\cite{ryckman-hca}.

\textit{Anxiety-driven competition avoidant} attitude is the exact opposite of hypercompetitive attitude.

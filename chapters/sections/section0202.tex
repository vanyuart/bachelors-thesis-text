%!TEX ROOT=main.tex

\section{Competitive attitude}\label{sec:competitive-attitude}

%https://www.frontiersin.org/articles/10.3389/fpsyg.2018.00779/full

%Regarding the personal-development competitive attitude (PDCA),
%the primary focus is on personal growth and on the enjoyment and mastery of the task in a competitive situation.
%The goal attainment and competition outcome (i.e., on winning) is important,
%but not at the expense of the derogation of other competitors (Ryckman et al., 1996).

Whereas the previous section created a basis for data management and data flow,
this section will introduce justification for competitiveness as an instrument for personal development.
Social engagement element of the application will be based on this justification.

In psychology competitiveness was considered as a unidimensional scale for a long time - up until 1990s.
The opposite of competitiveness was cooperation.
Cooperation could be seen as the opposite of competitiveness.
A research article "The Four Faces of Competition" summarized the development and research of recent decades
and introduced four-dimensional scale for competitiveness.\cite{the-four-faces-of-competetition}
This scale is heavily based on works of R.M.Ryckman.\cite{cit-ryckman-hca,cit-ryckman-adca,cit-ryckman-pdca}
Ryckman studied competitiveness as attitude rather than personal quality.
Overall there were identified four following attitudes regarding competition:
hypercompetitive, anxiety-driven competition avoidant, self-developmental, and lack of interest toward competition.

\textit{Hypercompetitive} attitude is represented by an aggressive approach to competitive situations.
One with hypercompetitive attitude has a goal of winning any competition they participate.
Winning could be gained even by unfair means.
This dominant attitude is destructive for personal relationships and considered psychologically unhealthy.
It is also associated with low self-actualization, low self-esteem, high aggression and dominance.\cite{cit-ryckman-hca}

\textit{Anxiety-driven competition avoidant} attitude is the opposite of hypercompetitive attitude.
A person with competition avoidant attitude will try to avoid entering competition by all means.
Their anxiety derives from the general process of competition rather than fear of loosing it.
Such attitude correlates with lower self-esteem and lower optimal psychological health overall.\cite{cit-ryckman-adca}

\textit{Self-developmental} attitude or the \textit{personal-development competitive attitude} (PDCA)
takes competition as an opportunity for personal development.
A person with dominant PDCA enjoys the process of competition, but in another way than one with dominant hypercompetitive
attitude.
For a PDCA dominant person, enjoyment comes from the process of skill improvement during a competition.
Winning also takes an important part, but not at the price of cheating or derogation of other competitors.
This trait is associated with higher self-esteem, high self-actualization, task enjoyment and self-discovery.\cite{cit-ryckman-pdca}

Lastly \textit{lack of interest toward competition} is different from the three above.
It simply describes disinterest in competitions and lack of any approach or avoidance.

These traits could be combined with each other, as not all of them are self-exclusive.
For example, one could have lack of interest as a primary trait and PDCA as a secondary.
A person with such traits would usually ignore competitive opportunities, but when engaged into one,
would use it as a possibility for personal growth and enjoy it.\cite{the-four-faces-of-competetition}

As it was revealed by the research, competition might be used as a tool of personal development.
Individuals with PDCA as primary or secondary trait might benefit from an application that will utilize their healthy
attitude toward competition.
Since competitions in the application will be created by users themselves and for their close social circle,
it could be helpful for ones with anxiety-driven competition avoidance.
Friendly competitions in the controlled environment might help them to overcome their anxiety.
Individuals with hypercompetitive attitude probably will not benefit from social part of application,
as they might take these competitions too seriously.
Although, the application still remains useful for them as a tool for personal development planning.
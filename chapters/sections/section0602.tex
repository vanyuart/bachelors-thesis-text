%!TEX ROOT=main.tex

\section{User testing}\label{sec:user-testing}

The goal of the user testing is to identify usability issues and critical application design flaws.
In given context, a critical design flaw would mean that some core features of the application
will make application unusable for a significant amount of users.
Five users were familiarized with the application under supervision using steps from the list below.

\begin{enumerate}
    \item Go to https://pda-web.herokuapp.com/
    \item Create an account
    \item Sign in to your account
    \item Create a habit
    \item Complete some habits for today
    \item Try to edit your habit (change days of week, times a day etc.)
    \item Create a goal
    \item Mark some goal tasks as finished
    \item Try to edit your goal (add tasks, change order etc.)
    \item Add a new friend (try typing username 'artem')
    \item Check your friend requests on "Friends" page
    \item After friend request accepted check "Activities" page for friend's activities
    \item Check "Statistics" page
    \item On "Statistics" page generate a habit to fill the graph
    \item Delete generated habit "Statistics test"
\end{enumerate}

After familiarization session users were left to use the application for the next week.
Some bugs related to streak reset and statistics calculation were found during user testing.
Below is the list of selected user notes order by priority from the most important.
Some was not included as they were caused by the lack of "Profile" page which was in the design, but was not yet implemented.

\begin{enumerate}
    \item \textbf{Edit progress from previous days.}\\
        This request will help to keep users engaged in case they forget to mark their progress.
        They will lose the streak and might as well lose interest in the application otherwise.
        It is possible to implement this feature with the current domain model, as the information about each habit fulfillment for each day is kept in "Habit daily progress" entity.
    \item \textbf{Leaderboards of habits fulfillment.}\\
        This request works well with the idea of social engagement.
        Deeper analysis of habits with shared progress between users is needed.
    \item \textbf{Groups of users.}\\
        This request is a natural development of the friend list idea.
        Users should be able to create groups where they could share habits, work on them together and see each other progress.
        The concept of leaderboards works well with this request.
    \item \textbf{Counter on habits.}\\
        Currently, when habit is repeated multiple times after marking one as done, the following will popup.
        Adding a counter that will show how many repeats are done and how many more to go will improve user experience.
        This request could be simply implemented with the current domain model.
    \item \textbf{There should be a feature of occasionally doing more repeats of a habit.}\\
        There is no simple way to occasionally mark more repeats on a habit.
        This request requires additional analysis before implementation.
\end{enumerate}

In conclusion, user testing helped to discover some bugs and gave several ideas to improve the application.
All users agreed on the fact that the main way of using this application would be via the mobile client.
No critical design flaws were identified during user testing.
Collected information would be valuable for the future implementation of the mobile client as well as improvement of current web client.
%!TEX ROOT=main.tex

\section{Deployment}\label{sec:deployment}

Solution was deployed to the Heroku cloud.\cite{heroku}
Applications there are deployed to "dynos".
Dyno on Heroku is a virtualized Linux container.
Containers provide isolated environment which brings several benefits.
The first is separation from the infrastructure.
This simplifies the process of deployment.
The second is the ease of scalability.
New containers might be deployed as needed to balance the incoming traffic.

Process of the deployment is straightforward.
Given that Heroku command-line interface is installed and user is authenticated.
First Heroku should be initialized in the project's root directory\@.

\begin{Verbatim}[frame=single]
heroku create
\end{Verbatim}

This will add new Git remote called "heroku" by default.
Deployment process will start simply by pushing code changes to the heroku remote.

\begin{Verbatim}[frame=single]
git push heroku master
\end{Verbatim}

The entire process might be monitored through server logs.

\begin{Verbatim}[frame=single]
heroku logs --tail -a $APP_NAME$
\end{Verbatim}

Deployed web client was available on https://pda-web.herokuapp.com/ at the time of writing.
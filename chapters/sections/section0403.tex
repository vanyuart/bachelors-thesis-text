%!TEX ROOT=main.tex

\section{Functional requirements}\label{sec:functional-requirements}

This section will introduce requirements regarding the application functionalities.

Requirements have their priority estimation.
High priority feature is a "must have" and necessarily needed in the application.
Medium priority feature is a "nice to have" and should be implemented for the best user experience.
Low priority feature is a "might have" which takes user experience level even further.

Complexity estimation is relative.
Medium might be considered as average time to implement a feature.
Low complexity level requirements takes noticeably shorter period of time to implement them.
High complexity level requirements as the opposite will take longer period of time.

\subsection{User registration and login}\label{subsec:user-registration-and-login}
\textbf{Complexity:} High\\
\textbf{Priority:} High\\
\textbf{Description:} In order to safely store and access data in the cloud, user needs registration and login features.\\
\textbf{Note:} This requirement brings the need for thorough security configuration.\\


\subsection{User login with a facebook account}\label{subsec:user-login-with-facebook-account}
\textbf{Complexity:} High\\
\textbf{Priority:} Medium\\
\textbf{Description:} Login through a Facebook account needed to minimize registration effort for a user.\\
\textbf{Note:} Meeting the Facebook's security criteria might take an additional effort.\\


\subsection{Friend list}\label{subsec:friend-list}
\textbf{Complexity:} Low\\
\textbf{Priority:} High\\
\textbf{Description:} Users need the ability to add each other to friend lists in order to share activities with each other.\\


\subsection{Application is usable without a registration}\label{subsec:application-is-usable-without-registration}
\textbf{Complexity:} High\\
\textbf{Priority:} Low\\
\textbf{Description:} Ability to try the application without a registration improves experience for a user.\\
\textbf{Note:} This requirement will take large amount of time for analysis and will increase overall complexity of the system.\\


\subsection{Habits management}\label{subsec:habits-management}
\textbf{Complexity:} Medium\\
\textbf{Priority:} High\\
\textbf{Description:} Users need an interface to manage their repetitive activities.\\


\subsection{Goals management}\label{subsec:goals-management}
\textbf{Complexity:} Medium\\
\textbf{Priority:} High\\
\textbf{Description:} Users need an interface to manager their long-term goals.\\


\subsection{Goals planning}\label{subsec:goals-planning}
\textbf{Complexity:} Medium\\
\textbf{Priority:} High\\
\textbf{Description:} Users need an interface to create and share goal plans.\\


\subsection{Visibility of habit, goal, goal plan}\label{subsec:visibility-for-habit-goal-goal-plan}
\textbf{Complexity:} Medium\\
\textbf{Priority:} High\\
\textbf{Description:} Users need an ability to change visibility of their data.
There will be three levels of visibility: private, selected friends only, all friends.\\


\subsection{Activity history}\label{subsec:activity-history}
\textbf{Complexity:} Medium\\
\textbf{Priority:} High\\
\textbf{Description:} Users need an interface to see their activities on goals and plans.\\


\subsection{Activity sharing}\label{subsec:activity-sharing}
\textbf{Complexity:} High\\
\textbf{Priority:} Medium\\
\textbf{Description:} Users need an interface to see activities of other users.\\
\textbf{Note:} This requirement brings the need for visibility management of goals and habits between users.\\


\subsection{Basic statistics display}\label{subsec:basic-statistic-display}
\textbf{Complexity:} Medium\\
\textbf{Priority:} Medium\\
\textbf{Description:} Users need an interface to see their statistics of following habits and fulfilling goals.\\


\subsection{Gamification elements by activity points}\label{subsec:gamification-elements-by-activity-points}
\textbf{Complexity:} Low\\
\textbf{Priority:} Low\\
\textbf{Description:} Each finished activity related to habit gives user one point.
Activity points displayed in user profile.\\
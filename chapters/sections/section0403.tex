%!TEX ROOT=main.tex

\section{Functional requirements}\label{sec:functional-requirements}

This section will introduced requirements regarding the application functionalities.

Requirements have their priority estimation.
High priority feature is a "must have" and necessarily needed in the application.
Medium priority feature is a "nice to have" and should be implemented for the best user experience.
Low priority feature is a "might have" which takes user experience level even further.

Complexity estimation is relative.
Medium might be considered as average time to implement a feature.
Low complexity level requirements takes noticeably shorter period of time to implement them.
High complexity level requirements as the opposite will take longer period of time.

\subsection{User registration and login}\label{subsec:user-registration-and-login}
\textbf{Complexity:} High\\
\textbf{Priority:} High\\
\textbf{Description:} In order to safely store and access users data in the cloud, user has register and login into the application.\\
\textbf{Note:} This requirement carries the need for thorough security configuration.


\subsection{User login with a facebook account}\label{subsec:user-login-with-facebook-account}
\textbf{Complexity:} High\\
\textbf{Priority:} Medium\\
\textbf{Description:} In order eliminate registration process for a user.
\textbf{Note:} Meeting the Facebook's security criteria might take additional effort.


\subsection{Application is usable without a registration}\label{subsec:application-is-usable-without-registration}
\textbf{Complexity:} High\\
\textbf{Priority:} Medium\\
\textbf{Description:} In order to give user chance to try it as simple as possible.
\textbf{Note:} This requirement will take large amount of time for analysis and will increase overall complexity of the system.


\subsection{Habits management}\label{subsec:habits-management}
\textbf{Complexity:} Medium\\
\textbf{Priority:} High\\
\textbf{Description:} In order to safely store and access users data in the cloud.
%!TEX ROOT=main.tex

\section{Data collection}\label{sec:data-collection}

Data collection was done using Google search engine as it was the most popular search engine in 2020 with over 90\% market share.\cite{google-market-share}
The first step was to find applications using following keywords: "personal development apps", "habit tracker", "self-improvement apps".
The top five results for each keyword were used to gain the initial dataset.
The next step was to manually filter gathered data.
The first step of filtration was to manually remove duplicates.
The second step was to remove activity specific applications: sport, meditation, brain training.
The third step was to remove applications with static textual content on topics of psychology, personal development and life improvement techniques.
The fourth step was to remove applications aimed exclusively at personal care, skincare, mood journals and sleep journals.
The fifth step was to remove business and productivity applications aimed primarily at small scale time management, short-term goals and daily productivity.
The sixth step was to remove applications aimed strongly at bad habit breaking, such as smoking and drinking.
The seventh step was to remove applications available for single platform exclusively, e.g., iOS\@.
The final step was to remove applications that are very similar in terms of interface, user experience and functionality.

\begin{table}[t!]
    \centering
    \begin{adjustbox}{width=\textwidth}
        \begin{ctucolortab}
            \begin{tabular}{llllll}
                \bfseries Action & \bfseries K1 & \bfseries K2 & \bfseries K3 & \bfseries Total results & \bfseries Difference \\\Midrule
                Initial dataset & 52 & 38 & 62  & 152 & +152\\
                Duplicates & 46 & 25 & 23  & 94 & -58\\
                Sport oriented & 41 & 23 & 22  & 86 & -8\\
                Meditation oriented & 38 & 19 & 19  & 76 & -10\\
                Brain training oriented & 25 & 16 & 18  & 59 & -17\\
                Topics oriented & 15 & 16 & 10  & 41 & -18\\
                Personal care oriented & 5 & 11 & 7  & 23 & -18\\
                Business and productivity oriented & 4 & 11 & 5  & 20 & -3\\
                Bad habits oriented & 4 & 10 & 1  & 15 & -5\\
                Single platform & 2 & 5 & 1  & 8 & -7\\
                Similar applications & 2 & 2 & 1  & 5 & -3\\
                \bottomrule
                \multicolumn{6}{1}{K1="Personal development apps", K2="Habit tracker", K3="Self-improvement apps"}
            \end{tabular}
        \end{ctucolortab}
    \end{adjustbox}
    \caption{Dataset filtering}
    \label{tab:dataset-filtering}
\end{table}

The result of filtering is 5 applications that will be analyzed in the next section.
Details of the dataset filtering process are described in Table~\ref{tab:dataset-filtering}.
\textit{Action} column contains the removal criteria during a filtration step.
\textit{K1} to \textit{K3} columns contain the amount of applications by a keyword.
\textit{Total results} column contains the sum of K1, K2 and K3 columns.
\textit{Difference} column contains the difference of total results against previous row.
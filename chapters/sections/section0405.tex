%!TEX ROOT=main.tex

\section{Technological stack}\label{sec:technological-stack}

Technological stack was considered with developer experience and free of use in mind.
Popularity, community and performance were also taken into account.

\subsection{Server side}\label{subsec:ts-server-side}

Kotlin with Spring framework was selected as the server side language and PostgreSQL as the database.\cite{kotlin, spring, postgresql}
Kotlin is a modern programming language that compiles to Java compatible bytecode.
This allows to use any Java library with Kotlin.

There are many reasons to use Kotlin over Java.
Some of them are null safety, final variables and classes.
This helps to write more robust code.
Boilerplate code is significantly reduced in comparison with Java.
Code written in Kotlin is shorter.

Following code snippet is written in Java.\\
\\
\begin{Verbatim}[frame=single]
public class Person {
    private String firstName;
    private String lastName;

    public Person(String firstName, String lastName) {
        this.firstName = firstName;
        this.lastName = lastName;
    }

    public String getFirstName() {
        return firstName;
    }

    public void setFirstName(String firstName) {
        this.firstName = firstName;
    }

    public String getLastName() {
        return lastName;
    }

    public void setLastName(String lastName) {
        this.lastName = lastName;
    }
}
\end{Verbatim}

Kotlin equivalent for that snippet would be the following.\\
\begin{Verbatim}[frame=single]
class Person(var firstName: String, var lastName: String)
\end{Verbatim}

Another benefit of Kotlin is the strong influence of functional programming,
resulting in convenient features such as scope functions, high order functions and lambda expressions.

In regard to downsides, Kotlin's clean builds take longer to compile.\cite{kotlin-compile}

\subsection{Frontend}\label{subsec:ts-frontend}

React was selected as a frontend framework with Material-UI as a design library.\cite{react, material}
React is developed by Facebook and positions itself as a UI library, which means further libraries
such as Axios (client-server communication) and Redux (state management) might be needed.\cite{axios, redux}
Material-UI is a library with React components built according to Google's UI design guidelines.
Combination of React with Material-UI is responsive out of the box, which means web client will be displayed
correctly on mobile devices without an extra amount of work.

Also, it is relatively less complicated to prototype a mobile client using React Native in the future.\cite{react-native}
React Native is a UI library that allows the creation of UI for Android and iOS mobile devices.
Code written in React Native is very similar to React code, some components could even be reused.
React Native code then transpiled to platform native UI code for Android or iOS\@.

TypeScript was also considered as frontend technology but idea was dismissed due to insufficient time resources.
Benefit of TypeScript is strong typing which JavaScript lacks.\cite{typescript}

\subsection{Deployment}\label{subsec:ts-deployment}

Heroku cloud platform was selected as a deployment option.\ref{heroku}
Simple, minimalistic interface and straightforward deployment via command-line interface are the benefits over Amazon AWS or Google Cloud.
Pricing is affordable only for proofs of concept, MVPs and small projects.
Heroku is not the best choice for large application or high throughput application as the pricing goes 2 to 3
times more expensive than with Amazon AWS.\cite{heroku-vs-aws}

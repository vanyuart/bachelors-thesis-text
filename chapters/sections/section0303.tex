%!TEX ROOT=main.tex
\section{Data analysis}\label{sec:data-analysis}

The following 5 applications will be tested and analyzed in this section: Uloo, Habitica, Habitify, TickTick, Habitshare.
%Remente and Coach.me.
Android and PlayMarket are going to be used as a testing platform.
Downloads, reviews and overall score are captured at the time of writing that is October 2020.

\subsection{Uloo}\label{subsec:uloo}

\begin{figure}[h!]
    \includegraphics[width=0.20\textwidth]{images/uloo-logo}
    \caption{Uloo logo~\cite{uloo-logo}}
    \label{fig:uloo-logo}
\end{figure}

Uloo is a goal pursuing application.
It has features for goals setting, progress tracking, communication between users, common and private goals tracking and paid mentoring by coaches.
%That multiplatform application covers all of the set criteria.
It is not really suitable for simple repetitive tasks.
Another demerit is that it is not usable without a paid premium subscription.
Functions are very limited in a free version.
Even with a premium subscription, it is not possible to create personalised goals.
A core feature, which is goal's progress tracking, demonstrated a breaking bug during the test of functionalities.
%Daily time cap added to goal progress regardless of actual spend time value

\begin{table}[h!]
    \centering
    \begin{adjustbox}{width=\textwidth}
        \begin{ctucolortab}
            \begin{tabular}{cc}
                \bfseries Pros & \bfseries Cons\\\Midrule
                Can start discovering without an account & Unusable without a paid premium\\
                Fair UI/UX & Only predefined goals\\
                Time or count as & Squad size limited to 8\\
                progress measurement & Small amount of \\
                Squads for goals pursuing in teams & installs and reviews\\
                Chat in squads & Bugs and unstable behaviour\\
            \end{tabular}
        \end{ctucolortab}
    \end{adjustbox}
    \caption{Uloo pros and cons}\label{tab:uloo-pros-cons}
\end{table}

Uloo has between 1,000--2,000 downloads and 28 reviews with an average score 4.2 out of 5 on PlayMarket.
%It is fair to say that Uloo is new on the market.
Uloo's implementation of progress sharing and squads might serve as an inspiration.
Limitation to only predefined goals is an example of not giving enough freedom to a user and should be avoided in the new application.
%Relatively expensive mandatory subscription and questionable quality, extremely limited experience with free version and small market share creates a space for another application.


\subsection{Habitica}\label{subsec:habitica}

\begin{figure}[h!]
    \includegraphics[width=0.20\textwidth]{images/habitica-logo}
    \caption{Habitica logo~\cite{habitica-logo}}
    \label{fig:habitica-logo}
\end{figure}

Habitica is an application for habits, daily goals and To-Do's management.
It has features for interaction between users.
Habitica is a great example of gamification.
Application is presented in a role-playing game style.
Users develop their game characters by executing daily chores and pursuing goals.
What Habitica lacks is long-term goal management.
Another potential problem is that the amount of gamification in the app would not work for many people,
especially those who are not interested in video games.

\begin{table}[h!]
    \centering
    \begin{ctucolortab}
        \begin{tabular}{cc}
            \bfseries Pros & \bfseries Cons\\\Midrule
            Great UI/UX & May be too much of gamification \\
            Many free features & for some users \\
            Gamification & Not suitable for long-term goals \\
            Social features & \\
        \end{tabular}
    \end{ctucolortab}
    \caption{Habitica pros and cons}\label{tab:habitica-pros-cons}
\end{table}

Habitica has over 1,000,000 downloads and 17,334 reviews with an average score of 4.4 out of 5 on PlayMarket.
Habitica is a well-developed application with a large user base.
Core idea of Habitica is to convert daily routines into a game, which is a feature that would not work for everyone.
Habitica's strong aim at gamification and overall UI makes it irrelevant in terms of the future application design.
On the other hand, it shows importance of possibility to interact between users which is a great part of Habitica.


\subsection{Habitify}\label{subsec:habitify}

\begin{figure}[h!]
    \includegraphics[width=0.20\textwidth]{images/habitify-logo}
    \caption{Habitify logo~\cite{habitify-logo}}
    \label{fig:habitify-logo}
\end{figure}

Habitify describes itself as a data-driven habit tracker.
It provides strong data visualization and activity planning features.
Another strong feature is synchronization with multiple devices such as smartwatches and integration with health applications.
Application does not provide social elements and UI might have been more interesting.
The biggest problem is a very limited functionality of the free version.

\begin{table}[h!]
    \centering
    \begin{ctucolortab}
        \begin{tabular}{cc}
            \bfseries Pros & \bfseries Cons\\\Midrule
            Data visualization & Social element is missing\\
            Devices synchronization & Unusable without a paid premium\\
            Health applications integration & Unengaging UI \\
        \end{tabular}
    \end{ctucolortab}
    \caption{Habitify pros and cons}\label{tab:habitify-pros-cons}
\end{table}

Habitify has over 100,000 downloads and 1,137 reviews with an average score 3.9 out of 5 on PlayMarket.
Data visualization in this application serves as a good inspiration.
On the other hand, UI is an example of minimalistic approach done wrong with a disengaging and empty feel overall.


\subsection{TickTick}\label{subsec:ticktick}

\begin{figure}[h!]
    \includegraphics[width=0.20\textwidth]{images/ticktick-logo}
    \caption{TickTick logo~\cite{ticktick-logo}}
    \label{fig:ticktick-logo}
\end{figure}

TickTick is an activity planner application.
It provides lists, reminders and calendars as well as tools for activity planning.
It is also possible to share lists between users.
TickTick was not primarily designed as a personal development application.
However, still might be used as one.
Loaded variety of fields and functions creates an exhausting user experience.

\begin{table}[h!]
    \centering
    \begin{ctucolortab}
        \begin{tabular}{cc}
            \bfseries Pros & \bfseries Cons\\\Midrule
            Usable without an account & Large amount of typing \\
            Shared lists & Too many options \\
            Good UI & for data management \\
             & Questionable UX \\
             & for personal development \\
        \end{tabular}
    \end{ctucolortab}
    \caption{TickTick pros and cons}\label{tab:ticktick-pros-cons}
\end{table}

TickTick has over 1,000,000 downloads and 67,692 reviews with an average score 4.6 out of 5 on PlayMarket.
Despite being a well-received application for activity planning, it is not the best option for personal development.
Giving too much freedom and functions to a user might decrease usability of an application and should be avoided.


\subsection{Habitshare}\label{subsec:habitshare}

\begin{figure}[h!]
    \includegraphics[width=0.20\textwidth]{images/habitshare-logo}
    \caption{Habitshare logo~\cite{habitshare-logo}}
    \label{fig:habitshare-logo}
\end{figure}

Habitshare describes itself a social habit tracker.
It utilizes user's social circle for extra accountability.
User creates habits they want to track and do so using calendar.
The application is only suitable for simple and repetitive activity tracking.

\begin{table}[h!]
    \centering
    \begin{ctucolortab}
        \begin{tabular}{cc}
            \bfseries Pros & \bfseries Cons\\\Midrule
            Highly social & Not usable without an account\\
            Usable without paying & Not suitable for long-term goals\\
        \end{tabular}
    \end{ctucolortab}
    \caption{Habitshare pros and cons}\label{tab:habitshare-pros-cons}
\end{table}

Habitshare has over 100,000 downloads and 658 reviews with an average score 4.5 out of 5 on PlayMarket.
Social part of application seems to be well accepted by users as well as simple and minimalistic UI\@.

%\subsection{Remente}\label{subsec:remente}
%
%Remente description
%
%\begin{table}[h!]
%    \centering
%    \begin{ctucolortab}
%        \begin{tabular}{cc}
%            \bfseries Pros & \bfseries Cons\\\Midrule
%            One & Two\\
%            Three & \\
%        \end{tabular}
%    \end{ctucolortab}
%    \caption{Habitify pros and cons.}\label{tab:remente-pros-cons}
%\end{table}
%
%Remente summary
%
%
%\subsection{Coach.me}\label{subsec:coachme}
%
%Coach.me description
%
%\begin{table}[h!]
%    \centering
%    \begin{ctucolortab}
%        \begin{tabular}{cc}
%            \bfseries Pros & \bfseries Cons\\\Midrule
%            One & Two\\
%            Three & \\
%        \end{tabular}
%    \end{ctucolortab}
%    \caption{Coach.me pros and cons.}\label{tab:coachme-pros-cons}
%\end{table}
%
%Coach.me summary


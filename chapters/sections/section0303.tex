%!TEX ROOT=main.tex
\section{Data analysis}\label{sec:data-analysis}

Following 7 applications will be tested and analyzed in this section: Uloo, Habitica, Habitify, TickTick, Habitshare, Remente and Coach.me.

\subsection{Uloo}\label{subsec:uloo}

\begin{figure}[h!]
    \includegraphics[width=0.20\textwidth]{images/uloo-logo.png}
    \caption{Uloo logo~\cite{uloo-logo}}
    \label{fig:uloo-logo}
\end{figure}

Uloo is a goal pursuing application.
Some of it features are goals setting, progress tracking, communication between users, common and private goals tracking and paid mentoring by coaches.
That multiplatform application covers all of the set criteria.
As for the downsides application is not usable without a paid premium subscription as all functions are very limited in a free version.
Even with a premium it is not possible to create own goals.
A core feature, that is goal progress tracking, demonstrated breaking bug during the test of functionalities.
%Daily time cap added to goal progress regardless of actual spend time value

\begin{table}[h!]
    \centering
    \begin{adjustbox}{width=\textwidth}
        \begin{ctucolortab}
            \begin{tabular}{cc}
                \bfseries Pros & \bfseries Cons\\\Midrule
                Can start discovering without an account & Unusable without a paid premium\\
                Fair UI/UX & Only predefined goals\\
                Time or count as a progress measurement & Squad size limited to 8\\
                Squads for goals pursuing in teams & Small amount of installs and reviews\\
                Chat in squads & Bugs and unstable behaviour\\
            \end{tabular}
        \end{ctucolortab}
    \end{adjustbox}
    \caption{Uloo pros and cons.}\label{tab:uloo-pros-cons}
\end{table}

At the time of writing Uloo had between 1,000--2,000 installs and 28 reviews with a score 4/5 on PlayMarket.
It is fair to say that Uloo is new on the market.
It has good features that should be captured in the future application.
Relatively expensive subscription, extremely limited experience with free version and small market share creates a space for another application.


\subsection{Habitica}\label{subsec:habitica}

\begin{figure}[h!]
    \includegraphics[width=0.20\textwidth]{images/habitica}
    \caption{Uloo logo~\cite{habitica-logo}}
    \label{fig:habitica-logo}
\end{figure}

Habitica is an application for habits, daily goals and To-Dos management.
It has features for interactions between users.
Habitica is a great example of gamification.
Application is presented in a role-playing game style.
Users develop their game characters by doing daily chores and pursuing goals.
What Habitica lacks is a long-term goals management.
Another potential problem is that amount of gamification in the app would not work for many people,
especially those who are not into video games.

\begin{table}[h!]
    \centering
    \begin{ctucolortab}
        \begin{tabular}{cc}
            \bfseries Pros & \bfseries Cons\\\Midrule
            Great UI/UX & May be too much of gamification for some\\
            Many free features & Not suitable for long-term goals\\
            Gamification & \\
        \end{tabular}
    \end{ctucolortab}
    \caption{Habitica pros and cons.}\label{tab:habitica-pros-cons}
\end{table}

At the time of writing Habitica had over 1,000,000 installs and 17,334 reviews with a score 4/5 on PlayMarket.
Habitica is a well developed application with a large user base.
Core idea of Habitica is to convert daily routine into a game, which is a feature that should not work for everyone.


\subsection{Habitify}\label{subsec:habitify}

Habitify description

\begin{table}[h!]
    \centering
    \begin{ctucolortab}
        \begin{tabular}{cc}
            \bfseries Pros & \bfseries Cons\\\Midrule
            One & Two\\
            Three & \\
        \end{tabular}
    \end{ctucolortab}
    \caption{Habitify pros and cons.}\label{tab:habitify-pros-cons}
\end{table}

Habitify summary


\subsection{TickTick}\label{subsec:ticktick}

TickTick description

\begin{table}[h!]
    \centering
    \begin{ctucolortab}
        \begin{tabular}{cc}
            \bfseries Pros & \bfseries Cons\\\Midrule
            Usable without an account & Two\\
            Three & \\
        \end{tabular}
    \end{ctucolortab}
    \caption{Habitify pros and cons.}\label{tab:tickTick-pros-cons}
\end{table}

TickTick summary


\subsection{Habitshare}\label{subsec:habitshare}

Habitshare description

\begin{table}[h!]
    \centering
    \begin{ctucolortab}
        \begin{tabular}{cc}
            \bfseries Pros & \bfseries Cons\\\Midrule
            One & Two\\
            Three & \\
        \end{tabular}
    \end{ctucolortab}
    \caption{Habitify pros and cons.}\label{tab:habitshare-pros-cons}
\end{table}

TickTick summary


\subsection{Remente}\label{subsec:remente}

Remente description

\begin{table}[h!]
    \centering
    \begin{ctucolortab}
        \begin{tabular}{cc}
            \bfseries Pros & \bfseries Cons\\\Midrule
            One & Two\\
            Three & \\
        \end{tabular}
    \end{ctucolortab}
    \caption{Habitify pros and cons.}\label{tab:remente-pros-cons}
\end{table}

Remente summary


\subsection{Coach.me}\label{subsec:coachme}

Coach.me description

\begin{table}[h!]
    \centering
    \begin{ctucolortab}
        \begin{tabular}{cc}
            \bfseries Pros & \bfseries Cons\\\Midrule
            One & Two\\
            Three & \\
        \end{tabular}
    \end{ctucolortab}
    \caption{Coach.me pros and cons.}\label{tab:coachme-pros-cons}
\end{table}

Coach.me summary


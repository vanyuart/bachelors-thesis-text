%!TEX ROOT=main.tex
\section{Data analysis}\label{sec:data-analysis}

Following 5 applications will be tested and analyzed in this section: Uloo, Habitica, Habitify, TickTick, Habitshare.
%Remente and Coach.me.
Android and PlayMarket is going to be used as a testing platform.

\subsection{Uloo}\label{subsec:uloo}

\begin{figure}[h!]
    \includegraphics[width=0.20\textwidth]{images/uloo-logo.png}
    \caption{Uloo logo.\cite{uloo-logo}}
    \label{fig:uloo-logo}
\end{figure}

Uloo is a goal pursuing application.
Some of it features are goals setting, progress tracking, communication between users, common and private goals tracking and paid mentoring by coaches.
%That multiplatform application covers all of the set criteria.
As for the downsides application is not usable without a paid premium subscription as all functions are very limited in a free version.
Even with a premium it is not possible to create own goals.
A core feature, that is goal progress tracking, demonstrated breaking bug during the test of functionalities.
%Daily time cap added to goal progress regardless of actual spend time value

\begin{table}[h!]
    \centering
    \begin{adjustbox}{width=\textwidth}
        \begin{ctucolortab}
            \begin{tabular}{cc}
                \bfseries Pros & \bfseries Cons\\\Midrule
                Can start discovering without an account & Unusable without a paid premium\\
                Fair UI/UX & Only predefined goals\\
                Time or count as & Squad size limited to 8\\
                progress measurement & Small amount of \\
                Squads for goals pursuing in teams & installs and reviews\\
                Chat in squads & Bugs and unstable behaviour\\
            \end{tabular}
        \end{ctucolortab}
    \end{adjustbox}
    \caption{Uloo pros and cons.}\label{tab:uloo-pros-cons}
\end{table}

At the time of writing Uloo had between 1,000--2,000 installs and 28 reviews with an average score 4.2 out of 5 on PlayMarket.
%It is fair to say that Uloo is new on the market.
Uloo's implementation of progress sharing and squads might serve as an inspiration.
%Relatively expensive mandatory subscription and questionable quality, extremely limited experience with free version and small market share creates a space for another application.


\subsection{Habitica}\label{subsec:habitica}

\begin{figure}[h!]
    \includegraphics[width=0.20\textwidth]{images/habitica-logo.png}
    \caption{Habitica logo.\cite{habitica-logo}}
    \label{fig:habitica-logo}
\end{figure}

Habitica is an application for habits, daily goals and To-Dos management.
It has features for interactions between users.
Habitica is a great example of gamification.
Application is presented in a role-playing game style.
Users develop their game characters by doing daily chores and pursuing goals.
What Habitica lacks is a long-term goals management.
Another potential problem is that amount of gamification in the app would not work for many people,
especially those who are not into video games.

\begin{table}[h!]
    \centering
    \begin{ctucolortab}
        \begin{tabular}{cc}
            \bfseries Pros & \bfseries Cons\\\Midrule
            Great UI/UX & May be too much of gamification \\
            Many free features & for some users \\
            Gamification & Not suitable for long-term goals \\
            Social features & \\
        \end{tabular}
    \end{ctucolortab}
    \caption{Habitica pros and cons.}\label{tab:habitica-pros-cons}
\end{table}

At the time of writing Habitica had over 1,000,000 installs and 17,334 reviews with an average score of 4.4 out of 5 on PlayMarket.
Habitica is a well developed application with a large user base.
Core idea of Habitica is to convert daily routine into a game, which is a feature that should not work for everyone.
Habitica's aim at gamification and overall UI makes it irrelevant in terms of the future application design.
On the other hand, it shows importance of possibility to interact between users which is a great part of Habitica.


\subsection{Habitify}\label{subsec:habitify}

\begin{figure}[h!]
    \includegraphics[width=0.20\textwidth]{images/habitify-logo.png}
    \caption{Habitify logo.\cite{habitify-logo}}
    \label{fig:habitify-logo}
\end{figure}

Habitify describes itself as a data-driven habit tracker.
It provides strong data visualization and activity planning features.
Another strong feature is synchronization with multiple devices as smartwatch and integration with health applications.
Application does not provide social elements and UI might have been more interesting.
The biggest problem is a very limited functionality of the free version.

\begin{table}[h!]
    \centering
    \begin{ctucolortab}
        \begin{tabular}{cc}
            \bfseries Pros & \bfseries Cons\\\Midrule
            Data visualization & Social element is missing\\
            Devices synchronization & Unusable without a paid premium\\
            Health applications integration & Unengaging UI \\
        \end{tabular}
    \end{ctucolortab}
    \caption{Habitify pros and cons.}\label{tab:habitify-pros-cons}
\end{table}

At the time of writing Habitify had over 100,000 installs and 1,137 reviews with an average score 3.9 out of 5 on PlayMarket.
Data visualization in this applications serves as a good inspiration.
On the other hand, UI is an example of minimalistic approach done wrong with an overall boring and empty feel.


\subsection{TickTick}\label{subsec:ticktick}

\begin{figure}[h!]
    \includegraphics[width=0.20\textwidth]{images/ticktick-logo.png}
    \caption{TickTick logo.\cite{ticktick-logo}}
    \label{fig:ticktick-logo}
\end{figure}

TickTick is an activity planner application.
It provides lists, reminders and calendars as tools for activity planning.
It is also possible to share lists between users.
TickTick does not designed as personal development application, but rather might be used as one.
Variety of functions and fields creates more work for a user in order to actually start the process of personal development.

\begin{table}[h!]
    \centering
    \begin{ctucolortab}
        \begin{tabular}{cc}
            \bfseries Pros & \bfseries Cons\\\Midrule
            Usable without an account & Large amount of typing \\
            Shared lists & Too many options \\
            Good UI & for data management \\
             & Questionable UX \\
             & for personal development \\
        \end{tabular}
    \end{ctucolortab}
    \caption{TickTick pros and cons.}\label{tab:ticktick-pros-cons}
\end{table}

At the time of writing TickTick had over 1,000,000 installs and 67,692 reviews with an average score 4.6 out of 5 on PlayMarket.
Despite being a well received application for activity planning, it is not the best option for personal development.
Giving too much freedom and functions to a user might severe usability of an application and should be avoided.


\subsection{Habitshare}\label{subsec:habitshare}

\begin{figure}[h!]
    \includegraphics[width=0.20\textwidth]{images/habitshare-logo.png}
    \caption{Habitshare logo.\cite{habitshare-logo}}
    \label{fig:habitshare-logo}
\end{figure}

Habitshare describes itself a social habit tracker.
It utilizes users social circle for extra accountability.


\begin{table}[h!]
    \centering
    \begin{ctucolortab}
        \begin{tabular}{cc}
            \bfseries Pros & \bfseries Cons\\\Midrule
            Highly social & Not usable without an account\\
            Usable without paying & Aimed at habits\\
            Usable without paying & \\
        \end{tabular}
    \end{ctucolortab}
    \caption{Habitify pros and cons.}\label{tab:habitshare-pros-cons}
\end{table}

At the time of writing Habitshare had over 100,000 installs and 658 reviews with an average score 4.5 out of 5 on PlayMarket.

%\subsection{Remente}\label{subsec:remente}
%
%Remente description
%
%\begin{table}[h!]
%    \centering
%    \begin{ctucolortab}
%        \begin{tabular}{cc}
%            \bfseries Pros & \bfseries Cons\\\Midrule
%            One & Two\\
%            Three & \\
%        \end{tabular}
%    \end{ctucolortab}
%    \caption{Habitify pros and cons.}\label{tab:remente-pros-cons}
%\end{table}
%
%Remente summary
%
%
%\subsection{Coach.me}\label{subsec:coachme}
%
%Coach.me description
%
%\begin{table}[h!]
%    \centering
%    \begin{ctucolortab}
%        \begin{tabular}{cc}
%            \bfseries Pros & \bfseries Cons\\\Midrule
%            One & Two\\
%            Three & \\
%        \end{tabular}
%    \end{ctucolortab}
%    \caption{Coach.me pros and cons.}\label{tab:coachme-pros-cons}
%\end{table}
%
%Coach.me summary


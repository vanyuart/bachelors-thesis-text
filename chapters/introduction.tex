%!TEX ROOT=main.tex

\chapter{Introduction}\label{ch:introduction}

{\color{red}//IDEA Friendly competition}

This chapter describes motivation, problem description and the basic idea
of the future application.
~\nameref{sec:motivation} section contains personal point of view of the author,
detailed analysis will be the part of the~\nameref{ch:personal-development-domain} chapter.
~\nameref{sec:problem-description} section describes what problem the future application will attempt to solve.
Detailed approach to 


\section{Motivation}\label{sec:motivation}

{\color{gray}//ASK Can I write here from my perspective?}

From an elementary school till the final year of university studies, one was able to compare himself to peers.
This could be the source of strong motivation that stimulates continuous process of learning.
In order to keep myself from falling into daily routine of work when my studies are over,
but instead to continue to actively study, I decided to look into applications for personal development.

Short research showed that there is a gap in the market that could be filled.
This paper will shortly examine the domain of personal development, compare existing solutions and provide an analysis for a
new application.
{\color{gray}//ASK Mention implementation here?}


The idea of friendly competition is also used by sports and games.

It is possible to apply this idea to many activities

Same principle could be applied to many activities such

In order to prevent myself from falling into daily working routine, I



Personal development could be measured in many {\color{blue}//TODO which?} ways.
In terms of this paper we will consider
Since the end of my bachelor's s
Bla la la.

\section{Problem description}\label{sec:problem-description}

\section{Possible solution}\label{sec:possible-solution}

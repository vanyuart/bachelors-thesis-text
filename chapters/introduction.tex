%!TEX ROOT=main.tex


\chapter{Introduction}\label{ch:introduction}

This chapter will introduce the reader to author's motivation, problem description and the basic idea of the future application.

~\nameref{sec:introduction-motivation} section describes personal point of view of the author.
Detailed analysis of the related domain will be the part of ~\nameref{ch:personal-development-domain} chapter.
~\nameref{sec:problem-description} section describes problems the application will attempt to solve and
~\nameref{sec:solution-idea} section will draw basic idea of the application, whereas exact approach
will be introduced in following chapters: ~\nameref{ch:solution-analysis}, ~\nameref{ch:implementation}, ~\nameref{ch:testing}.


\section{Motivation}\label{sec:introduction-motivation}

{\color{gray}//ASK Can I write here from my own perspective?}

From an elementary school till the final year of university studies, one was able to compare themselves with peers.
In specific cases this could be the source of strong motivation that stimulates the continuous process of learning.
In order to keep myself from falling into daily routine of work when my studies are over,
but instead to continue to actively study, I decided to look into applications for personal development.

Prompt research demonstrated that applications with social engagement elements tend to be well accepted by users,
but also revealed a shortage of such applications.
Those that compiled with the social engagement criteria lacked crucial functions
that could be beneficial for the process of personal development.

This paper will shortly examine the domain of personal development, compare existing solutions and provide an analysis for a
new application.
{\color{gray}//ASK Mention implementation here?}


\section{Problem description}\label{sec:problem-description}

Personal development is a life-long process.
As well as many other activities, this process is more effective with defined, measured approach.
There are many well-build solutions aimed to structure process of personal development.
These solutions have variety of ways to approach their users, which means that each could find an application that works for them.
The problem with these applications is a small to none amount of social engagement.
User could rely exclusively on their own motivation and discipline in order to use such application effectively and on a long-term basis.

Another type of personal development applications are habit trackers.
These applications aimed at development of new desired habits and abandonment of unwanted.
Habit trackers often include gamification elements and usually there is a slight social engagement element.
There are also habit trackers that use the idea of friendly competition, which introduces higher level of social engagement.
On the other hand, these applications usually lack more structural approach, visual reports and planning features.

Application that motivate user through interaction with others and capable of activity planning and data visualisation
will take the best from both application types mentioned above and could provide better user experience.
The problem of this paper is analysis and proposal ~{\color{gray}//ASK and implementation?} of such application.


\section{Solution idea}\label{sec:solution-idea}

In order to use the best parts of previously mentioned application types, two main aspects should be designed thoroughly.
The first is the ability of application to provide an effective way of data input and organization.
The second is the high but optional level of social engagement.

The fist objective is going to be achieved through the main entity of personal development process - goals.
Goals will provide an interface for storage and organization of targets that user would like to pursue.
A goal progress is going to be measured by user activities related the goal.
Those activities could be used for data visualisation and highlighting of user's the most prioritized goals.
The application should also provide a minimal interface for activity planning.

The second objective is to increase user's motivation through members of theirs social circle.
This is going to be achieved through the sharing of personal goals and shared goals.
The former is a goal created by user for themselves and shared with other on read-only basis.
The latter is a goal where more than one user could write their goal activities.
This separation will provide flexible and substantial social engagement element.
Optionality of described feature is going to be achieved through the definition of goal's scope.

The biggest challenge of the described application is the design of intuitive and minimalistic user interface.
This ends the basic description of the solution idea.

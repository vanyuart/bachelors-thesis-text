%!TEX ROOT=main.tex


\chapter{Introduction}\label{ch:introduction}

This chapter will introduce the reader to author's motivation, problem description and the basic idea of the future application.
This thesis pursues several goals.
The first is to analyze the domain of personal development and verify that competition might be used as a motivator.
The second is to analyze existing solutions against the criteria, based on the domain analysis.
The third is to implement and test the first version of a new application.
%~\nameref{sec:introduction-motivation} section describes personal point of view of the author.
%Detailed analysis of the related domain will be the part of ~\nameref{ch:personal-development-domain} chapter.
%~\nameref{sec:problem-description} section describes problems the application will attempt to solve and
%~\nameref{sec:solution-idea} section will draw basic idea of the application, whereas exact approach
%will be introduced in following chapters: ~\nameref{ch:solution-analysis}, ~\nameref{ch:implementation}, ~\nameref{ch:testing}.

\section{Motivation}\label{sec:introduction-motivation}

From elementary school to the final year of university studies, one was able to compare themselves with peers.
In some cases this could be the source of strong motivation that stimulates a continuous process of learning.
In order to keep myself from falling into a daily routine of work at the completion of my studies and to be able to continue to actively study, I decided to look
into applications for personal development.

Prompt research demonstrated that applications with social engagement elements tend to be well accepted by users
but also revealed a shortage of such applications.
Those that complied with the social engagement criterion lacked crucial functions that could be beneficial for the process of personal development.
Also, only a small amount of them used competitiveness as a motivation for personal development.

This paper will shortly examine the domain of personal development and justify that friendly competition might be used as a tool for personal development;
compare existing solutions and provide an analysis for a new application.


\section{Problem description}\label{sec:problem-description}

Personal development is an ongoing process.
As well as many other activities, this process is more effective with a defined and measured approach.
There are many well-built solutions aimed to structure the process of personal development.
These solutions have a variety of ways to approach their users thus user(s) could find an application that works for them.
The problem with these applications is the almost nonexistent utilization of social engagement.
Users could rely exclusively on their motivation and discipline in order to use such application(s) effectively and on a long-term basis.

Another type of personal development applications are habit trackers.
These applications aim at the development of newly desired habits and discarding redundant ones.
Habit trackers often include gamification elements and usually there is minimal social engagement elements.
There are also habit trackers that use the idea of friendly competition which introduces higher level of social engagement.
On the other hand, these applications usually lack a structural approach, visual reports and planning features.

A new application that motivates a user through the interaction with others, capable of long-term activity planning and repetitive habit management should be proposed.
This will take the best from both application types mentioned above and could provide better user experience.
The problem that this paper is trying to solve is an analysis and proposal of such application.

\section{Solution idea}\label{sec:solution-idea}

In order to use the best parts of previously mentioned application types, two main aspects should be designed thoroughly.
The first is the ability of the application to provide an effective way of data entry and management.
The second is the high but optional level of social engagement.

The fist objective, is going to be achieved by separation of activities on two types - goals and habits.
Goals will provide an interface for input and organization of targets that a user would like to pursue.
A goal's progress is going to be measured by completion of tasks.
Habits, on the other hand, will provide an interface for repetitive activities that do not require planning.

The second objective is to increase user's motivation through members of their social circle.
This is going to be achieved through activity sharing.
Optionality of this feature would be available through visibility setting in goals and habits.
This separation will provide a substantial amount of social engagement.

The biggest challenge of the described application is the design of an intuitive and minimalistic user interface.
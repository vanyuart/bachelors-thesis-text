%!TEX ROOT=main.tex

\chapter{Introduction}\label{ch:introduction}

{\color{red}//IDEA Friendly competition}

This chapter will introduce the reader to author's motivation, problem description and the basic idea of the future application.

~\nameref{sec:introduction-motivation} section describes personal point of view of the author.
Detailed analysis of the related domain will be the part of ~\nameref{ch:personal-development-domain} chapter.

~\nameref{sec:problem-description} section describes problems application will attempt to solve and
~\nameref{sec:solution-idea} section will draw basic idea of the application, whereas exact approach
will be introduced in following chapters: ~\nameref{ch:solution-analysis}, ~\nameref{ch:implementation}, ~\nameref{ch:testing}.


\section{Motivation}\label{sec:introduction-motivation}

{\color{gray}//ASK Can I write here from my perspective?}

From an elementary school till the final year of university studies, one was able to compare himself to peers.
This could be the source of strong motivation that stimulates continuous process of learning.
In order to keep myself from falling into daily routine of work when my studies are over,
but instead to continue to actively study, I decided to look into applications for personal development.
Prompt research showed that there is a lack of applications that could provide strong social engagement features
as well as be helpful without them.
Applications with social engagement elements seems to be well accepted by users.
Those that exists lack crucial functionalities that could help the process and aim exclusively to the competition idea.
This paper will shortly examine the domain of personal development, compare existing solutions and provide an analysis for a
new application.
{\color{gray}//ASK Mention implementation here?}

\section{Problem description}\label{sec:problem-description}

Personal development is a life-long process.
As well as many other activities, this process is more effective with defined, measured approach.
There are many well-build solutions aimed to structure process of personal development.
These solutions have variety of ways to approach their users, which means that each could find an application that works for them.
The problem with these applications is small to none amount of social engagement.
Users should be motivated and disciplined enough in order to use such application for a long time.

Another type of personal development applications are habit trackers.
These applications aimed at development of new desired habits and abandonment of unwanted.
Habit trackers often include gamification elements and usually there is a slight social engagement element.
On the other hand, these applications usually lack more structural approach, visual reports and planning features.

Application that motivate user through interaction with others and capable of activity planning and data visualisation
will take the best from both application types mentioned above and provide better user experience.
The problem of this paper is analysis and proposal ~{\color{gray}//ASK and implementation?} of such application.

\section{Solution idea}\label{sec:solution-idea}

%!TEX ROOT=main.tex

\chapter{Conclusion}\label{ch:conclusion}

The domain research provided an understanding into the personal development process.
This understanding allowed proposition of the data management.
The second part of the domain research justified that the competitiveness might be used as a motivator for personal development.

Analysis of existing solutions allowed detecting common pitfalls and gave ideas for features that might improve overall user experience.
This analysis also demonstrated uniqueness of the application idea and therefore proved valuable for further solution analysis and implementation.

Based on the research and analysis of existing solutions, the solution analysis was created.
Research of the domain served as the basis for domain model.
Functional requirements were mostly derived from the analysis of existing solutions.

Created analysis served as a basis for the first version of the application.
User testing verified created concepts and provided valid ideas for the future analysis and development.

The next step would be the analysis of information gathered from the prototype development and testing.
Deeper analysis of UI/UX part should be done.
Later, a prototype of the mobile client might be created.